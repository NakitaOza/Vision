\section{Related Work}
\label{sec:relatedwork}
There are some algorithm that can detect lines in real-time. Such as sidewalk, automobile's lane. Algorithms for drone and automobile are actively discussed. Such as color based road following algorithms and so on like \cite[Color-based road detection in urban traffic scenes]{he2004color}. \newline
In previous ones, there are mostly depends on colors of images. A benchmark road follower system, SCARF\cite[Supervised Classification Applied to Road Following]{seng2008sidewalk} also uses Gaussian color model too. However, There are many limitations in applying  algorithms in real world. So, we gave up using the road detection system using Color Histogram and utilized edge detection along with road shape model to detect road areas in an image. Adopting this system, we made a blueprint in our research. Contrary to their studies, we used line detection and linear regression to get reliable line in ours and accumulated the tendency lines and found the direction of user. \newline
We first had to apply some filter to reduce noise of the image. Tomasi\cite[Bilateral filtering for gray and color images]{tomas1998bilateral} has presented the Bilateral filter by way of reduce noise of the image. It is smoothing filter for images that is characterized by being non-linear, edge-preserving, and noise-reducing. G Deng \cite[An adaptive Gaussian filter for noise reduction and edge detection]{deng1993adaptive} has present an adaptive Gaussian filter for noise reduction and edge detection. In order to get more specific information of image, although Bilateral filter has more time complexity than Gaussian's, we adopted the previous one. Tomas has present the way to use this filter in RGB or other color model images.\newline
After applying filters, we need to find contours of the image. To detect the image's edge, we choose Canny detection method by J. Canny\cite[A computational approach to edge detection.]{canny1986computational} Canny detection has got a two threshold to find contours. By adjusting two weighting value, it find edges by Sobel edge detector. After that, it check each edge's magnitude is maximum. If it is maximum, it only apply it and connect with other edges. It has got low error ratio, no spurious response, and well localized algorithm. It's result has mainly affected by this parameter. But defining two threshold's value is not easy. So P Bao \cite[Canny edge detection enhancement by scale multiplication.]{bao2005canny} has suggested scale multiplication to enhance the quality of detection. In our algorithm, we resized the image frame and set value by some test.\newline
To adjust appropriate threshold, video frame's color is need to be equalized. Histogram Equalization(HE) method is well known algorithm to revise image. Shah \cite[A REVIEW ON IMAGE CONTRAST ENHANCEMENT TECHNIQUES USING HISTOGRAM EQUALIZATION]{shah2015review} arranged this method to use. We equalized the video frame before applying The filter. The image is well-ordered by resizing and equalizing method. There are some adaptive filters. But our work is real-time image processing and needed to be operated in less time complexity. So we just set the threshold as a constant value by evaluating before the program run. \newline
In order to detect sidewalks, we have to find some lines in the image. There are many algorithms that find images's line. But our algorithm is primarily based on the Probabilistic Hough Transform. Probabilistic Hough Transform(PHT) by Kiryati\cite[A probabilistic Hough transform]{kiryati1991probabilistic} and Progressive Hough Transform(PPHT) by Barinova \cite[On detection of multiple object instances using hough transforms]{barinova2012detection} are good way to detect line features in the image. It is a technique which can be used to isolate features of a particular shape within a image. By adopting this function to contours, we get line features of each image frame. 
Succeeding finding lines of image, we verified each line's reliability by some equations. \newline
Finally, we get two most reliable lines on each video frame and accumulated it. Subsequently, we do linear regression on stacked lines. Seber\cite[Linear Regression Analysis]{seber2012linear} has arranged some linear regression method. Such as Ordinary Least Squares(OLS), Generalized Least Squares(GLS), Total Least Squares(TLS) and so on. Despite the fact that GLS and TLS is more reliable for analyzing data, we chose OLS to get relatively less elapsed time to be real-time processing. It is popular and powerful estimator whose result line has to go through average of the points. We, so as to, adopted OLS estimator to apply regression in get better results.