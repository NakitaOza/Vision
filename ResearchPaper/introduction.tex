\section{Introduction}
\label{sec:intro}
As the technology has been renovated each day, more convenient life for ours are also required. In our society, there are some obstacles that visually impaired person have to be suffered in living in high dense urban. Walking in road, as a pedestrian, it's so hard to them to go straight with no visual guide to help recognizing sidewalk's circumstance. To solve this problem, there are some sidewalk detection method, one by Seng(Seng, John S., and Thomas J. Norrie. "Sidewalk following using color histograms." Journal of Computing Sciences in Colleges 23.6 (2008): 172-180.) and there are many paper about sidewalk detection. But they are extracting data based on color models of histogram and analysis that to get information. Thus, in the situation of color models won't work properly, such as existing some obstacles at sidewalk or image's color is not fixed, it's reliability has been damaged. So, we determined to develop sidewalk detection algorithms based on line detection method and set two main focus of the algorithm. 
\newline First is that the algorithm is need to be based on line detection. To cope every situation that occur in real world's sidewalk such as fallen leaves, i	disarranged blocks and so on, we constructed the sidewalk detecting algorithm's as linear based. We get the linear features of the image that has more reliability to be a sidewalk and analysis the most data to get more accuracy and less elapsed time. We reduced it's time complexity by combining brief algorithm's result which operated in each video frame and analysis by some statistical regression method. It's result are more accurate than just adopting single complicated function to each input video frame and more faster than just doing complicated algorithm to each video frames. So we made some brief formula that testify the line's reliability and features to find whether the lines' are worth to regard as sidewalk or not. 
\newline 
Second is that the algorithm should work in real-time, as designed for visually impaired person and used in walking. In order to cope with each situation, it required that it's elapsed time be less than user's walking time and it be no spurious response and have high accuracy rating in finding sidewalk result and result's trend. In case of some spurious response happened, users only to believe program and follow the way what said to be result of algorithms. So if the conclusion is not suited for actual sidewalk, it is going to be a big problem and the algorithms' performance will be discussed. Also, if sidewalk's tendency is trembled as every steps of user, it is hard to user to recognize what is actual following path, as there exist so many path as they walk. In short, it is necessary to reduce elapsed time as work in real-time and to improve algorithms accuracy not to confuse user by final result. 
\newline
So in order to achieve this two purpose, with the help of computer vision, we have developed a Walking Assistance Android Application(WAAA) for the visually impaired. In order to develop WAAA, therefore, we need algorithms that can detect sidewalks and extract the surrounding noise. At first, we planed to use color histogram analysis to detect sidewalk. We studied various algorithms concerning sidewalk detection, and we found a algorithm which based on color histogram analysis. And it works reliably on the ideal sidewalk situation, which there are only one pattern and color, also no unexpected noise. But it turned out that, in reality, it don’t work well. Because sidewalk can be covered partly with the shadow of trees, some kind of obstacles, other pedestrians or abnormal pattern and color. For this reason, we need to make another algorithms that don’t use color histogram as much as we can. Finally, we decided to make an algorithm based on line detection. Through this solution, we could solve limitations which are caused by color histogram.
\newline
In this paper, we focus on the line detection and tendency of real-time images. We have studied how visually impaired people are walking on the sidewalk and what can usually happen in real world. It means that there are many fallen leaves. There are some pedestrians and many colors of blocks. Besides, there might be some corners where people must turn the direction at sidewalks. So, we made algorithms to solve these problems by line detection.
\newline
In our method, by comparing the subsequent video frames with that of real-time's, we increase the reliability of direction. It has little to do with machine learning, but it looks for accumulated input data set and evaluates the actual direction of pedestrian wants just like machine learning does. The goal of this algorithm is to detect the direction of users and he or she should follow. It gives users the information on which ways are better to follow and how to do that. As this algorithm is applied to WAAA, visually impaired people can recognize the correct direction and can make a decision about his or her way.